\section{Related Work}
\label{sec:related_work}

In this section we briefly discuss state-of-the-art indexing methods for handling multidimensional point data.

\textbf{Range Search in Data Points.}
One of the most basic data operations is that of filtering data that qualify a query predicate.
The query is in the form of a multidimensional range $R$ and it says ``find me the data points in the database that belong to $R$''.
The main idea in state-of-the-art multidimensional indexes is to divide the whole space into subregions \cite{Samet:book}.

 
Our work follows the same high level principles, i.e., data split, but it is the first to introduce  an adaptive indexing mechanism driven by query predicates.  
In all previous work, the index is built in one step a priori and no queries may be processed until the index is ready. 
On the contrary, in our work, query processing and index building are interleaved, resulting in a drastically reduced data to query time.






\textbf{Multidimensional Adaptive Indexing.} Even though in this paper we follow the same philosophy as in database cracking, our work is the first to design an adaptive index for multidimensional data.% processing and range search queries during query time.
Contrary to working with unidimensional arrays as in column-store relational databases in the case of database cracking, our work is based on multidimensional arrays,  which are suited for multidimensional indexing, where we index more than one columns at a time.  
Columns that are not indexed, can still be aligned using sideways cracking \cite{DBLP:conf/sigmod/IdreosKM09} which has been proposed in order to handle late tuple reconstruction in an adaptive column-store.
The concepts that have appeared in past adaptive indexing work apply here as well.
For example, updates and concurrency control can be handled exactly as in \cite{DBLP:journals/pvldb/GraefeHIKM12, CrackingTransactions, DBLP:conf/sigmod/IdreosKM07}.
Contrary to state-of-the-art multidimensional indexing work, initialization cost is kept low, bringing the ability to query the data set much sooner.   
We show both the significant bottleneck faced by state-of-the-art indexing as we grow to large data,
as well as the drastic improvement that adaptive indexing brings.
